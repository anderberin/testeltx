\chapter*[Prefácio]{Prefácio}
\addcontentsline{toc}{chapter}{Prefácio}

Com a popularização dos dispositivos móveis hoje parte integrante da vida de grande parte da população, o mercado de tecnologia teve grande foco no desenvolvimento de aplicativos que pudessem auxiliar em atividades recorrentes das pessoas, tornando processos anteriormente manuais, automatizados; visando sempre a agilidade, segurança e conforto ao usuário.
Neste contexto, diversas aplicações surgiram e dentre elas os aplicativos de gerenciamento de contas tem se tornado um grande atrativo para o público, seja um aplicativo voltado para um gerenciamento de longo ou curto prazo, tais aplicativos visam eliminar a necessidade de papéis e/ou informações guardadas e processadas pelo nosso cérebro.
Entretanto, assim como outros aplicativos, muitas questões podem ser esquecidas ou tratadas com irresponsabilidade durante a construção da aplicação, dentre elas as questões relacionadas à própria Interação Humano-Computador (IHC), assunto altamente debatido e analisado na atual era da informação.
É neste cenário que o presente documento visa analisar dois aplicativos com a finalidade de gerência/compartilhamento de contas da mesa comumente usado entre amigos que saem em grupo a fim de fazer um “happy hour”. Ao final desta análise propomos uma nova aplicação que vise suprir as necessidades encontradas nos aplicativo analisados, sendo os dados coletados a partir de um grupo de usuários conforme descrito neste documento.